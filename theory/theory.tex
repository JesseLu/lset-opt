\documentclass{article}
\usepackage{amsmath}
\title{Theory for \texttt{lset-opt} package}
\author{Jesse Lu, \texttt{jesselu@stanford.edu}}
\begin{document}
\maketitle
\tableofcontents

\section{Definitions}
\begin{description}
    \item[grid] A two-dimensional $m \times n$ regularly-spaced set of points. The grid spacing is 1 in both $x$ and $y$ directions.
    \item[cell] A box with length and height equal to 1, centered at a grid point.
    \item[$\phi$] Level-set function defined on grid. The zero-crossing of $\phi$ defines the contour of the shapes on the grid.
    \item[$p$] Fractional-filling function. Values of $p$ range from -1 to 1 depending on the relative volume of each material found in its cell.
\end{description}

\section{Regularization}
Given a candidate level-set function, $\hat{\phi}$, a regularized $\phi$ is computed in the following way:
\begin{enumerate}
    \item If any element of $\hat{\phi}$, $\hat{\phi_i}$, is exactly equal to 0, then let $\hat{\phi}_i = \epsilon$ where $\epsilon$ is the smallest positive number available.
    \item For all $\hat{\phi}_i$ not adjacent to a boundary point, fix the corresponding $\phi_i$ as 
    \begin{equation}
        \phi_i = \text{sign}(\hat{\phi}_i) = 
        \begin{cases}
            -1& \text{if } \hat{\phi}_i < 0, \\
            +1& \text{if } \hat{\phi}_i > 0.
        \end{cases}
    \end{equation}
    \item To determine the remaining $\phi_i$, solve the following:
    \begin{align}
        \text{minimize} \quad & \| \phi - \text{sign}(\hat{\phi})\|^2 \\
        \text{subject to} \quad & A \phi = 0,
    \end{align}
    where $A$ is a matrix which fixes the boundary points by forcing the ratio of the two adjacent values of $\phi_i$ to remain the same. The motivation behind the minimization objective is simply to keep values of phi from being unnecessarily small or large.
\end{enumerate}

\section{Conversion to fractional-filling}
The fractional-filling, $p$, of each cell is calculated in the following way,
\begin{equation}
    p_0 = \text{sign}(\phi_0)  (-1 + 2(\gamma_1 + \gamma_2)(\gamma_3 + \gamma_4)),
\end{equation}
where the subscript 0 refers to the value at the current grid point, and the subscripts 1, 2, 3, and 4 refer to values to the left, right, up, and down of the current grid point. 

The values of $\gamma$ are determined by
\begin{equation}
    \gamma_i = 
    \begin{cases}
        \frac{\phi_0}{\phi_0 - \phi_i} & 
            \text{if } |\phi_0| < |\phi_i| \text{ and } 
            \text{sign}(\phi_0) \ne \text{sign}(\phi_i), \\
        0.5 & \text{otherwise}.
    \end{cases}
\end{equation}
for $i = 1,2,3,4$.

\section{Topology update}
The topology defined by $\phi$ can be dynamically updated by specifying a desired change in $p$, $\Delta \hat{p}$. A realizable change in $p$, $\Delta p$, is then computed by first finding the matrix $\partial p / \partial \phi$, 
\begin{equation}
    \frac{\partial p}{\partial \phi} = 
    \frac{\partial p}{\partial \gamma} \frac{\partial \gamma}{\partial \phi},
\end{equation}
where
\begin{equation}
    \frac{\partial p_0}{\partial \gamma_i} = 
    \begin{cases}
        2\text{sign}(\phi)(\gamma_3 + \gamma_4) & \text{for $i = 1,2$,} \\
        2\text{sign}(\phi)(\gamma_1 + \gamma_2) & \text{for $i = 3,4$.} 
    \end{cases}
\end{equation}
and for $i = 1,3$
\begin{equation}
    \frac{\partial \gamma_i}{\partial \phi_0} = 
    \begin{cases}
        \frac{-\phi_i}{(\phi_0 - \phi_i)^2} & 
            \text{if } |\phi_0| < |\phi_i| \text{ and } 
            \text{sign}(\phi_0) \ne \text{sign}(\phi_i), \\
        0 & \text{otherwise.}
    \end{cases}
\end{equation}
\begin{equation}
    \frac{\partial \gamma_i}{\partial \phi_i} = 
    \begin{cases}
        \frac{-\phi_0}{(\phi_0 - \phi_i)^2} & 
            \text{if } |\phi_0| < |\phi_i| \text{ and } 
            \text{sign}(\phi_0) \ne \text{sign}(\phi_i), \\
        0 & \text{otherwise.}
    \end{cases}
\end{equation}
and for $i = 2,4$
\begin{equation}
    \frac{\partial \gamma_i}{\partial \phi_0} = 
    \begin{cases}
        \frac{-\phi_i}{(\phi_0 - \phi_i)^2} & 
            \text{if } |\phi_0| \le |\phi_i| \text{ and } 
            \text{sign}(\phi_0) \ne \text{sign}(\phi_i), \\
        0 & \text{otherwise.}
    \end{cases}
\end{equation}
\begin{equation}
    \frac{\partial \gamma_i}{\partial \phi_i} = 
    \begin{cases}
        \frac{-\phi_0}{(\phi_0 - \phi_i)^2} & 
            \text{if } |\phi_0| \le |\phi_i| \text{ and } 
            \text{sign}(\phi_0) \ne \text{sign}(\phi_i), \\
        0 & \text{otherwise.}
    \end{cases}
\end{equation}

Once we have $\partial p / \partial \phi$ then we solve the following for $\Delta \phi$,
\begin{align}
    \text{minimize} \quad & \| \Delta \phi \|^2 \\
    \text{subject to} \quad & \Delta \hat{p} = 
        \frac{\partial p}{\partial \phi} \Delta \phi.
\end{align}

$p + \Delta p$ is then given by converting $\phi + \Delta\phi$ to fractional-filling. In general, there will be a large discrepancy between $p + \Delta p$ and $p + \Delta \hat{p}$ because only certain fractional-fillings correspond to valid topologies. However, for 
\begin{equation}
    \lim_{\|\Delta\hat{p}\| \to 0} \|\Delta p - \Delta\hat{p}\| = 0
\end{equation}
if $\Delta\hat{p} = 0$ wherever $p = -1, +1$. 

\section{Island nucleation}
When performing a topological update, the nucleation of islands can be performed. An island can be formed where $\phi_{0,1,2,3,4}$ all equal either -1 or +1, and where $\Delta\hat{p}_0$ is of opposite sign. In this case,
\begin{equation}
\phi_0 = \frac{\sqrt{\Delta p}}{\sqrt{\Delta p} - 2\sqrt{2}} \phi_i.
\end{equation}    
\end{document}
